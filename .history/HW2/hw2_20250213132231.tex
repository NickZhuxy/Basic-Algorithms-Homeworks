\documentclass[11pt]{article}
\usepackage{amsmath, amssymb, amsfonts}
\usepackage[margin=1in]{geometry}
\usepackage{fancyhdr}
\usepackage{tcolorbox}
\usepackage{enumerate}

\usepackage{algorithmic}

\newcommand{\zo}{\{0,1\}}


\begin{document}

    \setlength{\headheight}{26pt}
    \pagestyle{fancy}
    \fancyhead[C]{\textbf{Basic Algorithms (Section 5)}\\Spring 2025}
    \fancyhead[R]{HW2 (Due 2/13 23:59)\\ Instructor: Jiaxin Guan}
    \fancyfoot[C]{}
    \fancyfoot[R]{\thepage}
    \renewcommand{\headrulewidth}{0.4pt}
    \renewcommand{\footrulewidth}{0.4 pt}
    
    %%%% EDIT THIS PART 
    %Put your name and Net ID here
	\fancyhead[L]{Name: Nick Zhu  \\ Net ID: xz4687}
    %Write your collaborators' names here
    \fancyfoot[L]{Discussion Partners:}
    %%%%%
	
    
    %Problem 1A
    \begin{tcolorbox}[title={Problem 1A (Recurrence Relations — 3 Styles, 9 pts)}]
        Use induction to show the asymptotic run-time corresponding to the following recurrence relation is $O(\log^2(n))$. \textbf{Do not use the Master Theorem here.}
        \begin{center}
            $T(0) = T(1) = 1$. For all $n\geq 2$, $T(n) = T(\lceil\frac{n}{2}\rceil-1) + \log n$.
        \end{center}
    \end{tcolorbox}
    %Write your solution here!
    \section*{Solution}

    
        \textbf{Base Case:} For \( n=0 \) and \( n=1 \), we have \( T(0)=T(1)=1 \). 
        By choosing \( C \) sufficiently large or by considering the claim only for \( n\ge2 \), the base case holds.

        \textbf{Inductive Hypothesis:} Assume that for all \( k < n \) (with \( n \) sufficiently large) 
        there exists a constant \( C > \frac{1}{2} \) such that:
        \[
        T(k) \le C (\log k)^2.
        \]
        
        \textbf{Inductive Step:} Consider \( n \ge 2 \). Then:
        \[
        T(n) = T\Big(\Big\lceil\frac{n}{2}\Big\rceil-1\Big) + \log n.
        \]
        Note that:
        \[
        \Big\lceil\frac{n}{2}\Big\rceil - 1 \le \frac{n}{2}.
        \]
        Thus, by the inductive hypothesis,
        \[
        T(n) \le C \left(\log \frac{n}{2}\right)^2 + \log n.
        \]
        Since \( \log \frac{n}{2} = \log n - 1 \), we have:
        \[
        T(n) \le C (\log n - 1)^2 + \log n.
        \]
        Expanding the square yields:
        \[
        T(n) \le C\left[(\log n)^2 - 2\log n + 1\right] + \log n = C (\log n)^2 - 2C \log n + C + \log n.
        \]
        This can be rewritten as:
        \[
        T(n) \le C (\log n)^2 - (2C - 1)\log n + C.
        \]
        To ensure that \( T(n) \le C (\log n)^2 \), it suffices that:
        \[
        -(2C - 1)\log n + C \le 0.
        \]
        This inequality holds for all sufficiently large \( n \) provided that:
        \[
        2C - 1 > 0 \quad \Longleftrightarrow \quad C > \frac{1}{2}.
        \]
        Hence, by choosing any constant \( C > \frac{1}{2} \), the inductive step is verified.

        \textbf{Conclusion:}  
        By mathematical induction, for all sufficiently large \( n \),
        \[
        T(n) \le C (\log n)^2.
        \]
        Thus, \( T(n)=O((\log n)^2) \).
        
        \hfill\(\Box\)
    \newpage
    
    %Problem 1B
    \begin{tcolorbox}[title={Problem 1B (9 pts)}]
        Use a recursion tree to find the asymptotic run-time corresponding to the following recurrence relation. You may assume any fractional input to $T$ is the greatest integer less than it (e.g., $T(\frac{2n}{3}) = T(\lfloor\frac{2n}{3}\rfloor)$).
        \begin{center}
            $T(0) = T(1) = 1$. For all $n\geq 2$, $T(n) = T(\frac{2n}{3}) + T(\frac{n}{5}) + n$
        \end{center}
    \end{tcolorbox}
    %Write your solution here!
    
    
    
    \newpage
    
    %Problem 1C
    \begin{tcolorbox}[title={Problem 1C (12 pts)}]
        Find the asymptotic run-time corresponding to each of the following recurrence relations using the Master Theorem. For each, explain which case of the Master Theorem applies and why.
        \begin{enumerate}[(a)]
            \item $T(n) = 2T(\frac{n}{5}) + \sqrt{2n}$.
            \item $T(n) = 3T(\frac{n}{3}) + n/2$.
            \item $T(n) = 4T(\frac{n}{2}) + n\log n$.
        \end{enumerate}
    \end{tcolorbox}
    %Write your solution here!

    \newpage
    
    %Problem 2
    \begin{tcolorbox}[title={Problem 2 (Array Search, 30 pts)}]
        You are given an array of $n$ integers $a_1 < a_2 < \dots < a_n$. Give an $O(\log n)$ algorithm that outputs an index $i$ where $a_i = i$, or outputs $\bot$ if such $i$ does not exist. Justify the correctness and time complexity of your proposed algorithm.
    \end{tcolorbox}
    %Write your solution here!
    
    \newpage
    
    %Problem 3
    \begin{tcolorbox}[title={Problem 3 (Malfunctioning Phones, 40 pts)}]
        A manufacturer has a recall on a set of $n$ cell phones, some of which have a malfunction which makes them unreliable. The manufacturer has built a machine that allows a pair of phones to test each other's correctness. Let $C_1, C_2$ be a pair of phones. The machine $M$ runs in the following way:
        \begin{enumerate}
            \item $M(C_1, C_2) = 11$ if both phones say the other is working.
            \item $M(C_1, C_2) = 10$ if one phone says the other is working and one phone says the other is malfunctioning. 
            \item $M(C_1, C_2) = 00$ if both phones say the other is malfunctioning. 
        \end{enumerate}
        Remember that malfunctioning phones cannot be trusted, so they may lie, tell the truth, or throw out a random response. Working phones, on the other hand, can be assumed to know if the other phone is working or malfunctioning always. 
        \begin{enumerate}[(a)]
            \item Show that if you know at least one working phone, all other working phones can be found by using $O(n)$ queries to $M$.
            \item Assume the majority of the phones are working, i.e., there are greater than $n/2$ working phones. Give an algorithm that can find a working phone in $O(n)$ queries to $M$. Justify the correctness and time complexity of your proposed algorithm. [Hint: Start by explaining how to use $O(n)$ queries to reduce the problem size by a constant factor.]
            \item Assume the majority of the phones are malfunctioning, i.e., there are fewer than $n/2$ working phones. Is there still a procedure (using $M$) that is guaranteed to find a working phone? Give a brief justification.
        \end{enumerate}
    \end{tcolorbox}
    %Write your solution here!
    
\end{document}