 \documentclass[11pt]{article}
\usepackage{amsmath, amssymb, amsfonts, amsthm}
\usepackage[margin=1in]{geometry}
\usepackage{fancyhdr}
\usepackage{tcolorbox}
\usepackage{enumerate}

\usepackage{algorithmic}

\newcommand{\zo}{\{0,1\}}


\begin{document}

    \setlength{\headheight}{26pt}
    \pagestyle{fancy}
    \fancyhead[C]{\textbf{Basic Algorithms (Section 5)}\\Spring 2025}
    \fancyhead[R]{HW2 (Due 2/13 23:59)\\ Instructor: Jiaxin Guan}
    \fancyfoot[C]{}
    \fancyfoot[R]{\thepage}
    \renewcommand{\headrulewidth}{0.4pt}
    \renewcommand{\footrulewidth}{0.4 pt}
    
    %%%% EDIT THIS PART 
    %Put your name and Net ID here
	\fancyhead[L]{Name: Nick Zhu  \\ Net ID: xz4687}
    %Write your collaborators' names here
    \fancyfoot[L]{Discussion Partners:Jueying Zhu, Siqi Zhu, Xichun Fan}
    %%%%%
	
    
    %Problem 1A
    \begin{tcolorbox}[title={Problem 1A (Recurrence Relations — 3 Styles, 9 pts)}]
        Use induction to show the asymptotic run-time corresponding to the following recurrence relation is $O(\log^2(n))$. \textbf{Do not use the Master Theorem here.}
        \begin{center}
            $T(0) = T(1) = 1$. For all $n\geq 2$, $T(n) = T(\lceil\frac{n}{2}\rceil-1) + \log n$.
        \end{center}
    \end{tcolorbox}
    %Write your solution here!
    \section*{Solution}

        We will use induction to show that \( T(n)=O((\log n)^2) \).

        \textbf{Base Case:}  
        For \( n = 0 \) and \( n = 1 \), we have:
        \[
        T(0)=T(1)=1.
        \]
        This satisfies the bound \( T(n) \le C (\log n)^2 \) for any constant \( C \) 
        (for these small values, the statement is vacuously true or we choose \( C \) large enough to cover these cases).

        \textbf{Inductive Hypothesis:}  
        Assume that for all \( k < n \) there exists a constant \( C > \frac{1}{2} \) such that:
        \[
        T(k) \le C (\log k)^2.
        \]

        \textbf{Inductive Step:}  
        For \( n \ge 2 \), we have:
        \[
        T(n) = T\Big(\Big\lceil\frac{n}{2}\Big\rceil-1\Big) + \log n.
        \]
        Since:
        \[
        \Big\lceil\frac{n}{2}\Big\rceil - 1 \le \frac{n}{2},
        \]
        the inductive hypothesis implies:
        \[
        T(n) \le C \left(\log \frac{n}{2}\right)^2 + \log n.
        \]
        since \( \log \frac{n}{2} = \log n - 1 \), we have:
        \[
        T(n) \le C (\log n - 1)^2 + \log n.
        \]
        Expanding the inequality, we have:
        \[
        T(n) \le C\left[(\log n)^2 - 2\log n + 1\right] + \log n = C (\log n)^2 - 2C \log n + C + \log n.
        \]
        and this simplifies to:
        \[
        T(n) \le C (\log n)^2 - (2C - 1)\log n + C.
        \]
        For the inequality \( T(n) \le C (\log n)^2 \) to hold, it is sufficient that:
        \[
        -(2C - 1)\log n + C \le 0.
        \]
        This inequality is satisfied for all sufficiently large \( n \) provided:
        \[
        2C - 1 > 0 \quad \Longleftrightarrow \quad C > \frac{1}{2}.
        \]
        Thus, by choosing any constant \( C > \frac{1}{2} \), the inductive step is verified.

        \textbf{Conclusion:}  
        By mathematical induction, for all sufficiently large \( n \),
        \[
        T(n) \le C (\log n)^2.
        \]
        Hence, we conclude that \( T(n)=O((\log n)^2) \).

        \hfill\(\Box\)
    \newpage
    
    %Problem 1B
    \begin{tcolorbox}[title={Problem 1B (9 pts)}]
        Use a recursion tree to find the asymptotic run-time corresponding to the following recurrence relation. You may assume any fractional input to $T$ is the greatest integer less than it (e.g., $T(\frac{2n}{3}) = T(\lfloor\frac{2n}{3}\rfloor)$).
        \begin{center}
            $T(0) = T(1) = 1$. For all $n\geq 2$, $T(n) = T(\frac{2n}{3}) + T(\frac{n}{5}) + n$
        \end{center}
    \end{tcolorbox}
    %Write your solution here!
    \section*{Solution}

    Given the recurrence, we will first draw the recursion tree, then use it to sum all the costs to reach the conclusion.

    \textbf{Step 1: Constructing the Recursion Tree}

    \begin{itemize}
        \item \textbf{Level 0 (Root):} The cost is \( n \).
        \item \textbf{Level 1:} The recurrence splits into two subproblems of sizes \(\frac{2n}{3}\) and \(\frac{n}{5}\). 
        Therefore, the cost at this level is:
        \[
        \frac{2n}{3} + \frac{n}{5} = \frac{13n}{15}.
        \]
        \item \textbf{Level \(i\):} At any node with subproblem size \( x \), the work done is proportional to \( x \). 
        Since each node produces subproblems of sizes \(\frac{2}{3}x\) and \(\frac{1}{5}x\), the sum of the subproblem sizes is:
        \[
        \frac{2}{3}x+\frac{1}{5}x = \frac{13}{15}x.
        \]
        Hence, if the total cost at level \(i\) is \( n\left(\frac{13}{15}\right)^i \), then the total cost at level \(i+1\) is:
        \[
        n\left(\frac{13}{15}\right)^{i+1}.
        \]
    \end{itemize}

    \textbf{Step 2: Summing the Costs}

    The total work is the sum over all levels until the subproblems become constant (which happens after \( \Theta(\log n) \) levels):
    \[
    T(n) = n + \frac{13n}{15} + \left(\frac{13}{15}\right)^2 n + \cdots.
    \]
    This is a geometric series with common ratio \( \frac{13}{15} < 1 \). Thus, the sum is:
    \[
    T(n) \le n \sum_{i=0}^{\infty}\left(\frac{13}{15}\right)^i
    = n \cdot \frac{1}{1-\frac{13}{15}}
    = n \cdot \frac{1}{\frac{2}{15}}
    = \frac{15}{2}n.
    \]
    Therefore,
    \[
    T(n) = \Theta(n).
    \]
    
    \newpage
    
    %Problem 1C
    \begin{tcolorbox}[title={Problem 1C (12 pts)}]
        Find the asymptotic run-time corresponding to each of the following recurrence relations using the Master Theorem. For each, explain which case of the Master Theorem applies and why.
        \begin{enumerate}[(a)]
            \item $T(n) = 2T(\frac{n}{5}) + \sqrt{2n}$.
            \item $T(n) = 3T(\frac{n}{3}) + n/2$.
            \item $T(n) = 4T(\frac{n}{2}) + n\log n$.
        \end{enumerate}
    \end{tcolorbox}
    %Write your solution here!
    \section*{Solution}

    \hspace*{2em}\textbf{(a) } Consider the recurrence 
    \[
    T(n)=2T\left(\frac{n}{5}\right)+\sqrt{2n}.
    \]
    We have:
    \[
    a=2,\quad b=5,\quad f(n)=\sqrt{2n}=\Theta(\sqrt{n}).
    \]
    \[
    n^{\log_b a}=n^{\log_5 2}=\Theta\left(n^{\log_5 2}\right).
    \]
    Since \(\log_5 2\approx 0.4307\),
    \[
    n^{\log_5 2} = \Theta\left(n^{0.4307}\right).
    \]
    Comparing \(f(n)=\Theta\left(n^{0.5}\right)\) with \(n^{0.4307}\), we see that
    \[
    f(n)=\Omega\left(n^{\log_5 2+\epsilon}\right) \quad \text{for } \epsilon\approx 0.0693>0.
    \]
    Moreover, the regularity condition holds since
    \[
    a\, f\left(\frac{n}{b}\right)=2\,\Theta\left(\sqrt{\frac{n}{5}}\right)=\Theta\left(\sqrt{n}\right)
    \]
    with a constant factor \(2/\sqrt{5}<1\). Hence, by Case 3 of the Master Theorem,
    \[
    T(n)=\Theta\left(f(n)\right)=\Theta\left(\sqrt{n}\right).
    \]

    \bigskip

    \textbf{(b) } Consider the recurrence 
    \[
    T(n)=3T\left(\frac{n}{3}\right)+\frac{n}{2}.
    \]
    We have:
    \[
    a=3,\quad b=3,\quad f(n)=\frac{n}{2}=\Theta(n).
    \]
    \[
    n^{\log_b a}=n^{\log_3 3}=n.
    \]
    Since \(f(n)=\Theta(n)=\Theta\left(n^{\log_3 3}\right)\), this corresponds to Case 2 of the Master Theorem. Thus,
    \[
    T(n)=\Theta(n\log n).
    \]

    \bigskip

    \textbf{(c) } Consider the recurrence 
    \[
    T(n)=4T\left(\frac{n}{2}\right)+n\log n.
    \]
    We have:
    \[
    a=4,\quad b=2,\quad f(n)=n\log n.
    \]
    \[
    n^{\log_b a}=n^{\log_2 4}=n^2.
    \]
    Since \(n\log n = o(n^2)\), we can write \(f(n)=O\big(n^{2-\epsilon}\big)\) for some \(0<\epsilon<1\) 
    (for instance, \(\epsilon=\frac{1}{2}\)). Therefore, by Case 1 of the Master Theorem,
    \[
    T(n)=\Theta(n^2).
    \]
    \newpage
    
    %Problem 2
    \begin{tcolorbox}[title={Problem 2 (Array Search, 30 pts)}]
        You are given an array of $n$ integers $a_1 < a_2 < \dots < a_n$. Give an $O(\log n)$ algorithm that outputs an index $i$ where $a_i = i$, or outputs $\bot$ if such $i$ does not exist. Justify the correctness and time complexity of your proposed algorithm.
    \end{tcolorbox}
    %Write your solution here!
    \section*{Solution}

    Define the function
    \[
    f(i) = a_i - i.
    \]
    Since the array is strictly increasing, \(f(i)\) is non-decreasing. We then perform a binary search for an index \(i\) with \(f(i)=0\) (i.e., \(a_i=i\)).

    \par\noindent\rule{\textwidth}{0.4pt}
    \smallskip        
    \textsc{FindFixedPoint}$(A, n)$
    \begin{algorithmic}[1]
        \STATE $L \gets 1$
        \STATE $R \gets n$
        \WHILE{$L \le R$}
            \STATE $mid \gets \lfloor (L+R)/2 \rfloor$
            \IF{$A[mid] = mid$}
                \RETURN $mid$
            \ELSIF{$A[mid] > mid$}
                \STATE $R \gets mid - 1$
            \ELSE
                \STATE $L \gets mid + 1$
            \ENDIF
        \ENDWHILE
        \RETURN $\bot$
    \end{algorithmic}
    \vspace{-2mm}
    \par\noindent\rule{\textwidth}{0.4pt}
    
    Now we justify the correctness and time complexity of my proposed FindFixedPoint algorithm.

    \textbf{Correctness:}  
    Since \(f(i)=a_i-i\) is non-decreasing, if \(f(mid) < 0\) then for all \(i < mid\) we have \(f(i) \leq f(mid) < 0\). 
    Similarly, if \(f(mid) > 0\) then for all \(i > mid\), \(f(i) \geq f(mid) > 0\). 
    Therefore, if there is an index \(i\) such that \(f(i)=0\), the binary search will locate it.
    \smallskip

    \textbf{Time Complexity:}  
    Each iteration of the loop reduces the size of the search interval by at least half. 
    Since the initial interval is of size \( n \), the number of iterations is \( O(\log n) \). 
    Each iteration does a constant amount of work, so the overall time complexity is \( O(\log n) \).
    
    \newpage
    
    %Problem 3
    \begin{tcolorbox}[title={Problem 3 (Malfunctioning Phones, 40 pts)}]
        A manufacturer has a recall on a set of $n$ cell phones, some of which have a malfunction which makes them unreliable. The manufacturer has built a machine that allows a pair of phones to test each other's correctness. Let $C_1, C_2$ be a pair of phones. The machine $M$ runs in the following way:
        \begin{enumerate}
            \item $M(C_1, C_2) = 11$ if both phones say the other is working.
            \item $M(C_1, C_2) = 10$ if one phone says the other is working and one phone says the other is malfunctioning. 
            \item $M(C_1, C_2) = 00$ if both phones say the other is malfunctioning. 
        \end{enumerate}
        Remember that malfunctioning phones cannot be trusted, so they may lie, tell the truth, or throw out a random response. Working phones, on the other hand, can be assumed to know if the other phone is working or malfunctioning always. 
        \begin{enumerate}[(a)]
            \item Show that if you know at least one working phone, all other working phones can be found by using $O(n)$ queries to $M$.
            \item Assume the majority of the phones are working, i.e., there are greater than $n/2$ working phones. Give an algorithm that can find a working phone in $O(n)$ queries to $M$. Justify the correctness and time complexity of your proposed algorithm. [Hint: Start by explaining how to use $O(n)$ queries to reduce the problem size by a constant factor.]
            \item Assume the majority of the phones are malfunctioning, i.e., there are fewer than $n/2$ working phones. Is there still a procedure (using $M$) that is guaranteed to find a working phone? Give a brief justification.
        \end{enumerate}
    \end{tcolorbox}
    %Write your solution here!
       
    \section*{Solution}
    \smallskip        

    \hspace*{1em}\textbf{(a) }
    
    Suppose we know  phone \(W\) is working. Since working phones always report correctly, we can determine whether any other phone \(X\) is working by querying the machine \(M(W,X)\). In particular, when \(W\) tests \(X\):
    \begin{itemize}
        \item If \(W\) reports that \(X\) is working (i.e. the output is either \(11\) or, in a mixed response, the trusted answer from \(W\) is that \(X\) is working), then \(X\) must be working.
        \item If \(W\) reports that \(X\) is malfunctioning, then \(X\) is malfunctioning.
    \end{itemize}
    Thus, by making one query for each of the other \(n-1\) phones, we can identify all working phones using \(O(n)\) queries.
    
    \bigskip
    
    \textbf{(b) }
     
    Since more than half of the \(n\) phones are working, we can use a pairwise elimination process to reduce the problem size while preserving at least one working phone.
    
    \medskip
    
    \begin{enumerate}
        \item \textbf{Pairing:} Partition the current set \(S\) ( all \(n\) phones) arbitrarily into pairs. If there is an odd one out, let it pass to the next round unpaired.
        \item \textbf{Testing:} For each pair \((C_i, C_j)\), run the query \(M(C_i,C_j)\).  
        \begin{itemize}
            \item If the machine returns \(11\) (both phones claim the other is working), then \emph{at least one} of them is working. (In fact, if both are working, the response is \(11\); if one is malfunctioning, the working phone will still correctly report the other as malfunctioning—but then the response would not be \(11\). Thus, a response of \(11\) strongly suggests that both are working. However, even if a pair of malfunctioning phones might collude to output \(11\), they are in the minority.)
            \item Otherwise (i.e. if the result is \(10\) or \(00\)), discard both phones.
        \end{itemize}
        \item \textbf{Recursion:} Let \(S'\) be the set of phones retained from all pairs. Since more than half of the phones are working, the majority property ensures that at least one working phone remains in \(S'\). Repeat the process on \(S'\) until only one candidate \(C^*\) remains.
    \end{enumerate}
    
    \textbf{Verification:}  
    Once a candidate \(C^*\) is obtained, use the procedure from part (a): Test \(C^*\) against every phone (or a known working phone, if already identified) to verify that \(C^*\) is working. This requires an additional \(O(n)\) queries.
    
    \medskip
    
    \underline{Correctness:}  
    We maintain the invariant that if there is a working phone in the current set \(S\), then after pairing and elimination at least one working phone remains in \(S'\). This is because:
    \begin{itemize}
        \item A pair of working phones always produces the outcome \(11\) when tested (since both are reliable).
        \item In a pair containing one working and one malfunctioning phone, the working phone will correctly report the malfunctioning one, so the outcome will not be \(11\), leading to discarding both. However, since working phones form a strict majority, such losses are limited.
        \item Recall the professor’s hint: if \(A > B\) and \(C \le D\), then \(A - C > B - D\). In our context, let \(A\) be the number of working phones and \(B\) the number of malfunctioning phones. Since working phones form a majority, we have \(A > B\). During the elimination process, when pairing phones, suppose we discard \(C\) working phones and \(D\) malfunctioning phones. The hint then guarantees that as long as \(C \le D\) (that is, we never eliminate more working phones than malfunctioning ones), the net surplus remains positive:
        \[
        A - C > B - D.
        \]
        This positive surplus ensures that after each elimination round, at least one working phone remains in the candidate set. Thus, the invariant that "if there is a working phone originally, then at least one working phone survives in the current set" is maintained throughout the algorithm. 
    \end{itemize}
    Thus, the candidate \(C^*\) produced by the elimination process is guaranteed to be working. 
    
    \medskip
    
    \underline{Time Complexity:}  
    Each pairing round makes \(O(n)\) queries and reduces the set size by at least a constant factor (roughly half). 
    Since there are \(O(1)\) rounds (in fact, \(O(\log n)\) rounds, but the constant factors add up to \(O(n)\) overall), 
    and the final verification step takes \(O(n)\) queries, the total number of queries is \(O(n)\).
    
    \bigskip
    
    \textbf{(c) }
    
    When fewer than \(n/2\) phones are working, the working phones are in the minority. 
    In this case, no procedure using the machine \(M\) is guaranteed to find a working phone. 
    The reason is that malfunctioning phones can provide arbitrary, even adversarial, responses. 
    They may create the responses of working phones (e.g., always outputting \(11\)) and thus hide the existence of any working phone. 
    Without a reliable majority, any elimination strategy can be subverted by the faulty phones’ unpredictable behavior.
    
\end{document}