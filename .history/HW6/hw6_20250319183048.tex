\documentclass[11pt]{article}
\usepackage{amsmath, amssymb, amsfonts}
\usepackage[margin=1in]{geometry}
\usepackage{fancyhdr}
\usepackage{tcolorbox}
\usepackage{enumerate}
\usepackage{tikz}

\usepackage{algorithmic}

\newcommand{\zo}{\{0,1\}}
\usetikzlibrary{shapes.geometric,arrows,fit,matrix,positioning}
\tikzset
{
    treenode/.style = {circle, draw=black, align=center, 
                          minimum size=1cm, anchor=center}
}

\begin{document}

    \setlength{\headheight}{26pt}
    \pagestyle{fancy}
    \fancyhead[C]{\textbf{Basic Algorithms (Section 5)}\\Spring 2025}
    \fancyhead[R]{HW6 (Due 3/20 23:59)\\ Instructor: Jiaxin Guan}
    \fancyfoot[C]{}
    \fancyfoot[R]{\thepage}
    \renewcommand{\headrulewidth}{0.4pt}
    \renewcommand{\footrulewidth}{0.4pt}
    
    %%%% EDIT THIS PART 
    %Put your name and Net ID here
	\fancyhead[L]{Name:  \\ Net ID: }
    %Write your collaborators' names here
    \fancyfoot[L]{Discussion Partners:}
    %%%%%

    %Problem 1
    \begin{tcolorbox}[title={Problem 1 (Knapsack, Take II, 50 pts)}] \setlength\parindent{1em}
    
     Alice and Bob are siblings who share a collection of items. There are \(n\) items, and each item \(i\) has a size \(s_i\), as well as separate valuations for Alice \(v_i^A\) and for Bob \(v_i^B\). Each sibling has their own knapsack with a capacity \(S\). Alice and Bob can pick disjoint subsets of items to maximize their combined valuation, with the constraint that the total size of the items they pick does not exceed the capacity of their respective knapsacks. Notice their subsets of items are disjoint because they cannot both pack the same item.

        \medskip
        
        Put formally, we want to maximize $\sum_{i\in I_A} v_i^A+\sum_{i\in I_B} v_i^B$ subject to the constraints $\sum_{i\in I_A} s_i\leq S$ and $\sum_{i\in I_B} s_i\leq S$. $I_A, I_B\subseteq\{1,2,\ldots, n\}$ represent the subsets of items that Alice and Bob each pick, and we require $I_A$ and $I_B$ to be disjoint ($I_A\cap I_B=\phi$). For simplicity, we only need to output the maximum combined valuation for Alice and Bob, not the subsets. 

        \begin{enumerate}[(a)]
            \item Find the max combined value Alice and Bob can get for the following input:
            \begin{itemize}
                \item \( n = 3 \)
                \item Item 1: \( s_1 = 2, v_1^A = 3, v_1^B = 5 \)
                \item Item 2: \( s_2 = 1, v_2^A = 4, v_2^B = 2 \)
                \item Item 3: \( s_3 = 3, v_3^A = 7, v_3^B = 6 \)
                \item Capacity of knapsacks \(S = 4\).
            \end{itemize} 
            \item Give a DP algorithm to solve this variant of the knapsack problem in $O(nS^2)$. Justify the correctness and runtime of your proposed algorithm. 
            \item How would you adapt your algorithm in part (b) if Alice and Bob have different knapsack capacities, \(S_A\) and \(S_B\), respectively? And how will the runtime of your algorithm change? You do not need to justify correctness for this part.
        \end{enumerate}
    \end{tcolorbox}
    %Write your solution here!

    \textbf{(a)}
    \[
    \begin{array}{rcl}
    \text{Item 1} &\to& \text{Bob (} s_1=2,\; v_1^B=5\text{)},\\[1mm]
    \text{Item 2} &\to& \text{Alice (} s_2=1,\; v_2^A=4\text{)},\\[1mm]
    \text{Item 3} &\to& \text{Alice (} s_3=3,\; v_3^A=7\text{)}.
    \end{array}
    \]
    Alice uses capacity \(1+3=4\) and Bob uses capacity \(2\le 4\), so the total combined value is 
    \[
    5+4+7=\boxed{16}.
    \]
    
    \medskip
    
    \textbf{(b)}
    
    \begin{enumerate}
        \item \textbf{Subproblem:}\\
        We define 
        \[
        K(i,a,b) = \text{the maximum combined valuation obtainable from items } i,i+1,\dots,n,
        \]
        given that Alice has remaining capacity \(a\) and Bob has remaining capacity \(b\) (with \(0\le a,b\le S\)). 
        The solution is given by \(K(1,S,S)\).
    
        \item \textbf{Guess:}\\
        For each item \(i\), we consider three possibilities:
        \begin{enumerate}
            \item Do not select item \(i\): use \(K(i+1,a,b)\).
            \item If \(a\ge s_i\), assign item \(i\) to Alice: obtain value \(v_i^A\) and reduce her capacity to \(a-s_i\), i.e., \(v_i^A+K(i+1,a-s_i,b)\).
            \item If \(b\ge s_i\), assign item \(i\) to Bob: obtain value \(v_i^B\) and reduce his capacity to \(b-s_i\), i.e., \(v_i^B+K(i+1,a,b-s_i)\).
        \end{enumerate}
    
        \item \textbf{Recurrence:}\\[1mm]
        \[
        K(i,a,b)=
        \begin{cases}
        0, & \text{if } i=n+1,\\[1mm]
        \max \Bigl\{ K(i+1,a,b), \; 
        \mathbf{1}_{\{a\ge s_i\}}\Bigl(v_i^A+K(i+1,a-s_i,b)\Bigr), \; 
        \mathbf{1}_{\{b\ge s_i\}}\Bigl(v_i^B+K(i+1,a,b-s_i)\Bigr) \Bigr\}, & \text{if } 1\le i\le n.
        \end{cases}
        \]
        Here, \(\mathbf{1}_{\{a\ge s_i\}}\) (and similarly for \(b\)) indicates that the corresponding option is considered only if the condition holds.
    
        \item We use a memorization / bottom-up approach. 
        
            \textbf{Time Complexity :}
            There are \((n+1)(S+1)(S+1)=O(nS^2)\) subproblems, and each is computed in \(O(1)\) time. 

            \textbf{Correctness :}

            \medskip
            
            \textbf{Optimal Substructure:} \\
            Consider the subproblem \(K(i,a,b)\), which represents the maximum combined valuation obtainable from items \(i, i+1, \dots, n\) given remaining capacities \(a\) for Alice and \(b\) for Bob. In an optimal solution for \(K(i,a,b)\), one of the following must occur:
            \begin{enumerate}
                \item Item \(i\) is not chosen. Then the optimal solution is exactly \(K(i+1,a,b)\).
                \item Item \(i\) is assigned to Alice (assuming \(a \ge s_i\)). Then the combined valuation is \(v_i^A\) plus an optimal solution for the remaining items with reduced capacity, namely \(K(i+1,a-s_i,b)\).
                \item Item \(i\) is assigned to Bob (assuming \(b \ge s_i\)). Then the combined valuation is \(v_i^B\) plus an optimal solution for the remaining items with reduced capacity, namely \(K(i+1,a,b-s_i)\).
            \end{enumerate}
            Since in each case the solution for the subproblem \(K(i,a,b)\) relies on an optimal solution to a smaller subproblem, the problem exhibits the optimal substructure property.
            
            \medskip
            
            \textbf{Correct Base Case:} \\
            When \(i = n+1\), no items remain to be considered. Hence, regardless of the remaining capacities \(a\) and \(b\), the maximum combined valuation is zero:
            \[
            K(n+1,a,b) = 0 \quad \text{for all } 0 \le a \le S \text{ and } 0 \le b \le S.
            \]
            This correctly reflects that there is no value to be gained when no items are available.
        
        \item \textbf{All set.}
    \end{enumerate}
    
    \medskip
    
    \textbf{(c)} \emph{Adaptation for Different Capacities \(S_A\) and \(S_B\):}
    
    \begin{enumerate}
        \item \textbf{Subproblem:}\\
        Redefine 
        \[
        K(i,a,b)= \text{the maximum combined valuation obtainable from items } i,i+1,\dots,n,
        \]
        where now \(0\le a\le S_A\) is the remaining capacity in Alice's knapsack and \(0\le b\le S_B\) is the remaining capacity in Bob's knapsack. Our goal becomes computing \(K(1,S_A,S_B)\).
    
        \item \textbf{Guess:}\\
        The choices remain the same:
        \begin{enumerate}
            \item Do not select item \(i\): \(K(i+1,a,b)\).
            \item If \(a\ge s_i\), assign item \(i\) to Alice: \(v_i^A+K(i+1,a-s_i,b)\).
            \item If \(b\ge s_i\), assign item \(i\) to Bob: \(v_i^B+K(i+1,a,b-s_i)\).
        \end{enumerate}
    
        \item \textbf{Recurrence:}\\[1mm]
        \[
        K(i,a,b)=
        \begin{cases}
        0, & \text{if } i=n+1,\\[1mm]
        \max \Bigl\{ K(i+1,a,b), \; 
        \mathbf{1}_{\{a\ge s_i\}}\Bigl(v_i^A+K(i+1,a-s_i,b)\Bigr), \; 
        \mathbf{1}_{\{b\ge s_i\}}\Bigl(v_i^B+K(i+1,a,b-s_i)\Bigr) \Bigr\}, & \text{if } 1\le i\le n.
        \end{cases}
        \]
    
        \item \textbf{Bottom-Up Approach and Analysis:}\\[1mm]
        We use a \textbf{bottom-up approach with memorization} to fill the DP table. The number of subproblems is \((n+1)(S_A+1)(S_B+1)=O(nS_A S_B)\) with each subproblem computed in constant time.
    
        \item \textbf{All set.}
    \end{enumerate}
    \newpage

    %Problem 2
    \begin{tcolorbox}[title={Problem 2 (Make Change, 50 points)}] \setlength\parindent{1em}
        
     Given a set of coin denominations $S=\{s_1, s_2,\dots, s_n\}$, consider the problem of making change for $m$ cents using the fewest
        number of coins. Assume that the set of coin denominations consists of only positive integers and that for each denomination you can use an unlimited number of them. For example, if $S=\{1, 5, 10\}$ and $m=17$, we can make the change of $17$ cents by taking one $10$, one $5$, and two $1$'s, or by taking seventeen $1$'s, but the former is preferred, as it uses a minimal number of coins ($4$ in this case).

    \begin{enumerate}[(a)]
    \item Let $S=\{1, 5, 10, 25\}$, and $m=83$, what is the minimal number of coins to make the change? Also, give the number of coins for each denomination used.
    \item Let $S=\{1, 7, 10, 15\}$, and $m=36$, what is the minimal number of coins to make the change? Also, give the number of coins for each denomination used.
    \item Give a DP algorithm to solve the make change problem in $O(nm)$ time. Your algorithm only needs to output the minimal number of coins, not the detailed allocation. You may assume the denominations always include $1$, so there is always a way to make the change. Justify the correctness and runtime of your proposed algorithm. \textit{(Hint: Let $DP[i,j]$ denote the optimal solution for making a change of $j$ cents using denominations $\{s_1, s_2, \dots, s_i\}$)}
    \end{enumerate}
    \end{tcolorbox}
    
    %Write your solution here!
    
    \newpage

    
    
\end{document}